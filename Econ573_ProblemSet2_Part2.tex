% Options for packages loaded elsewhere
\PassOptionsToPackage{unicode}{hyperref}
\PassOptionsToPackage{hyphens}{url}
%
\documentclass[
]{article}
\usepackage{amsmath,amssymb}
\usepackage{iftex}
\ifPDFTeX
  \usepackage[T1]{fontenc}
  \usepackage[utf8]{inputenc}
  \usepackage{textcomp} % provide euro and other symbols
\else % if luatex or xetex
  \usepackage{unicode-math} % this also loads fontspec
  \defaultfontfeatures{Scale=MatchLowercase}
  \defaultfontfeatures[\rmfamily]{Ligatures=TeX,Scale=1}
\fi
\usepackage{lmodern}
\ifPDFTeX\else
  % xetex/luatex font selection
\fi
% Use upquote if available, for straight quotes in verbatim environments
\IfFileExists{upquote.sty}{\usepackage{upquote}}{}
\IfFileExists{microtype.sty}{% use microtype if available
  \usepackage[]{microtype}
  \UseMicrotypeSet[protrusion]{basicmath} % disable protrusion for tt fonts
}{}
\makeatletter
\@ifundefined{KOMAClassName}{% if non-KOMA class
  \IfFileExists{parskip.sty}{%
    \usepackage{parskip}
  }{% else
    \setlength{\parindent}{0pt}
    \setlength{\parskip}{6pt plus 2pt minus 1pt}}
}{% if KOMA class
  \KOMAoptions{parskip=half}}
\makeatother
\usepackage{xcolor}
\usepackage[margin=1in]{geometry}
\usepackage{color}
\usepackage{fancyvrb}
\newcommand{\VerbBar}{|}
\newcommand{\VERB}{\Verb[commandchars=\\\{\}]}
\DefineVerbatimEnvironment{Highlighting}{Verbatim}{commandchars=\\\{\}}
% Add ',fontsize=\small' for more characters per line
\usepackage{framed}
\definecolor{shadecolor}{RGB}{248,248,248}
\newenvironment{Shaded}{\begin{snugshade}}{\end{snugshade}}
\newcommand{\AlertTok}[1]{\textcolor[rgb]{0.94,0.16,0.16}{#1}}
\newcommand{\AnnotationTok}[1]{\textcolor[rgb]{0.56,0.35,0.01}{\textbf{\textit{#1}}}}
\newcommand{\AttributeTok}[1]{\textcolor[rgb]{0.13,0.29,0.53}{#1}}
\newcommand{\BaseNTok}[1]{\textcolor[rgb]{0.00,0.00,0.81}{#1}}
\newcommand{\BuiltInTok}[1]{#1}
\newcommand{\CharTok}[1]{\textcolor[rgb]{0.31,0.60,0.02}{#1}}
\newcommand{\CommentTok}[1]{\textcolor[rgb]{0.56,0.35,0.01}{\textit{#1}}}
\newcommand{\CommentVarTok}[1]{\textcolor[rgb]{0.56,0.35,0.01}{\textbf{\textit{#1}}}}
\newcommand{\ConstantTok}[1]{\textcolor[rgb]{0.56,0.35,0.01}{#1}}
\newcommand{\ControlFlowTok}[1]{\textcolor[rgb]{0.13,0.29,0.53}{\textbf{#1}}}
\newcommand{\DataTypeTok}[1]{\textcolor[rgb]{0.13,0.29,0.53}{#1}}
\newcommand{\DecValTok}[1]{\textcolor[rgb]{0.00,0.00,0.81}{#1}}
\newcommand{\DocumentationTok}[1]{\textcolor[rgb]{0.56,0.35,0.01}{\textbf{\textit{#1}}}}
\newcommand{\ErrorTok}[1]{\textcolor[rgb]{0.64,0.00,0.00}{\textbf{#1}}}
\newcommand{\ExtensionTok}[1]{#1}
\newcommand{\FloatTok}[1]{\textcolor[rgb]{0.00,0.00,0.81}{#1}}
\newcommand{\FunctionTok}[1]{\textcolor[rgb]{0.13,0.29,0.53}{\textbf{#1}}}
\newcommand{\ImportTok}[1]{#1}
\newcommand{\InformationTok}[1]{\textcolor[rgb]{0.56,0.35,0.01}{\textbf{\textit{#1}}}}
\newcommand{\KeywordTok}[1]{\textcolor[rgb]{0.13,0.29,0.53}{\textbf{#1}}}
\newcommand{\NormalTok}[1]{#1}
\newcommand{\OperatorTok}[1]{\textcolor[rgb]{0.81,0.36,0.00}{\textbf{#1}}}
\newcommand{\OtherTok}[1]{\textcolor[rgb]{0.56,0.35,0.01}{#1}}
\newcommand{\PreprocessorTok}[1]{\textcolor[rgb]{0.56,0.35,0.01}{\textit{#1}}}
\newcommand{\RegionMarkerTok}[1]{#1}
\newcommand{\SpecialCharTok}[1]{\textcolor[rgb]{0.81,0.36,0.00}{\textbf{#1}}}
\newcommand{\SpecialStringTok}[1]{\textcolor[rgb]{0.31,0.60,0.02}{#1}}
\newcommand{\StringTok}[1]{\textcolor[rgb]{0.31,0.60,0.02}{#1}}
\newcommand{\VariableTok}[1]{\textcolor[rgb]{0.00,0.00,0.00}{#1}}
\newcommand{\VerbatimStringTok}[1]{\textcolor[rgb]{0.31,0.60,0.02}{#1}}
\newcommand{\WarningTok}[1]{\textcolor[rgb]{0.56,0.35,0.01}{\textbf{\textit{#1}}}}
\usepackage{graphicx}
\makeatletter
\def\maxwidth{\ifdim\Gin@nat@width>\linewidth\linewidth\else\Gin@nat@width\fi}
\def\maxheight{\ifdim\Gin@nat@height>\textheight\textheight\else\Gin@nat@height\fi}
\makeatother
% Scale images if necessary, so that they will not overflow the page
% margins by default, and it is still possible to overwrite the defaults
% using explicit options in \includegraphics[width, height, ...]{}
\setkeys{Gin}{width=\maxwidth,height=\maxheight,keepaspectratio}
% Set default figure placement to htbp
\makeatletter
\def\fps@figure{htbp}
\makeatother
\setlength{\emergencystretch}{3em} % prevent overfull lines
\providecommand{\tightlist}{%
  \setlength{\itemsep}{0pt}\setlength{\parskip}{0pt}}
\setcounter{secnumdepth}{-\maxdimen} % remove section numbering
\ifLuaTeX
  \usepackage{selnolig}  % disable illegal ligatures
\fi
\usepackage{bookmark}
\IfFileExists{xurl.sty}{\usepackage{xurl}}{} % add URL line breaks if available
\urlstyle{same}
\hypersetup{
  pdftitle={Econ 573: Problem Set 2 - Part 2},
  hidelinks,
  pdfcreator={LaTeX via pandoc}}

\title{Econ 573: Problem Set 2 - Part 2}
\author{}
\date{\vspace{-2.5em}}

\begin{document}
\maketitle

\begin{Shaded}
\begin{Highlighting}[]
\NormalTok{Auto }\OtherTok{=} \FunctionTok{read.csv}\NormalTok{(}\StringTok{"C:/Users/mateo/OneDrive {-} University of North Carolina at Chapel Hill/Courses/Spring 2025/Econ 573/Auto.csv"}\NormalTok{)}
\end{Highlighting}
\end{Shaded}

\emph{Question 8}

\emph{(a) Use the lm() function to perform a simple linear regression
with mpg as the response and horsepower as the predictor. Use the
summary() function to print the results. Comment on the output.}

\begin{Shaded}
\begin{Highlighting}[]
\NormalTok{Auto}\SpecialCharTok{$}\NormalTok{horsepower }\OtherTok{\textless{}{-}} \FunctionTok{as.numeric}\NormalTok{(}\FunctionTok{as.character}\NormalTok{(Auto}\SpecialCharTok{$}\NormalTok{horsepower)) }\CommentTok{\#Convert the horsepower variable form categorical to numeric}
\end{Highlighting}
\end{Shaded}

\begin{verbatim}
## Warning: NAs introducidos por coerción
\end{verbatim}

\begin{Shaded}
\begin{Highlighting}[]
\NormalTok{lmmpg }\OtherTok{=} \FunctionTok{lm}\NormalTok{(mpg}\SpecialCharTok{\textasciitilde{}}\NormalTok{horsepower, }\AttributeTok{data =}\NormalTok{ Auto) }\CommentTok{\#Perform the simple linear regression}
\FunctionTok{summary}\NormalTok{(lmmpg)}
\end{Highlighting}
\end{Shaded}

\begin{verbatim}
## 
## Call:
## lm(formula = mpg ~ horsepower, data = Auto)
## 
## Residuals:
##      Min       1Q   Median       3Q      Max 
## -13.5710  -3.2592  -0.3435   2.7630  16.9240 
## 
## Coefficients:
##              Estimate Std. Error t value Pr(>|t|)    
## (Intercept) 39.935861   0.717499   55.66   <2e-16 ***
## horsepower  -0.157845   0.006446  -24.49   <2e-16 ***
## ---
## Signif. codes:  0 '***' 0.001 '**' 0.01 '*' 0.05 '.' 0.1 ' ' 1
## 
## Residual standard error: 4.906 on 390 degrees of freedom
##   (5 observations deleted due to missingness)
## Multiple R-squared:  0.6059, Adjusted R-squared:  0.6049 
## F-statistic: 599.7 on 1 and 390 DF,  p-value: < 2.2e-16
\end{verbatim}

There seems to be a small negative relationship between the amount of
miles per gallon for a car and its horsepower. From its p-value, which
is very close to \(0\), we can infer strong evidence for this
relationship such that it is statistically significant. The predicted
coefficient for horsepower,\(\beta_1 = -0.157845\), tells us that every
one-unit increase in horsepower decreases mpg by -0.157845.

\begin{Shaded}
\begin{Highlighting}[]
\NormalTok{horsepower98 }\OtherTok{=} \FunctionTok{data.frame}\NormalTok{(}\AttributeTok{horsepower =} \DecValTok{98}\NormalTok{)}
\FunctionTok{predict}\NormalTok{(lmmpg, horsepower98, }\AttributeTok{interval =} \StringTok{"confidence"}\NormalTok{) }\CommentTok{\#Average mpg for cars with 98 horsepower}
\end{Highlighting}
\end{Shaded}

\begin{verbatim}
##        fit      lwr      upr
## 1 24.46708 23.97308 24.96108
\end{verbatim}

\begin{Shaded}
\begin{Highlighting}[]
\FunctionTok{predict}\NormalTok{(lmmpg, horsepower98, }\AttributeTok{interval =}  \StringTok{"prediction"}\NormalTok{) }\CommentTok{\#Predicted mpg for a randomly selected car with 98 horsepower}
\end{Highlighting}
\end{Shaded}

\begin{verbatim}
##        fit     lwr      upr
## 1 24.46708 14.8094 34.12476
\end{verbatim}

From these results we can interpret that we are 95\% confident that the
mean mpg for cars with 98 horsepower is between 23.80 and 25.40. For an
individual car with 98 horsepower, we predict with 95\% confidence that
its mpg will be between 20.50 and 28.34.

\emph{Question 9}

\emph{(a) Produce a scatterplot matrix which includes all of the
variables in the data set.}

\begin{Shaded}
\begin{Highlighting}[]
\FunctionTok{pairs}\NormalTok{(Auto[, }\FunctionTok{sapply}\NormalTok{(Auto, is.numeric)]) }\CommentTok{\#Creates a scatterplot matrix for all the variables, selecting only numeric columns}
\end{Highlighting}
\end{Shaded}

\includegraphics{Econ573_ProblemSet2_Part2_files/figure-latex/unnamed-chunk-4-1.pdf}

\emph{(b) Compute the matrix of correlations between the variables using
the function cor(). You will need to exclude the name variable, cor()
which is qualitative.}

\begin{Shaded}
\begin{Highlighting}[]
\FunctionTok{cor}\NormalTok{(Auto[, }\FunctionTok{sapply}\NormalTok{(Auto, is.numeric)]) }\CommentTok{\#Creates the matrix of correlations between variables, excluding the name variable.}
\end{Highlighting}
\end{Shaded}

\begin{verbatim}
##                     mpg  cylinders displacement horsepower     weight
## mpg           1.0000000 -0.7762599   -0.8044430         NA -0.8317389
## cylinders    -0.7762599  1.0000000    0.9509199         NA  0.8970169
## displacement -0.8044430  0.9509199    1.0000000         NA  0.9331044
## horsepower           NA         NA           NA          1         NA
## weight       -0.8317389  0.8970169    0.9331044         NA  1.0000000
## acceleration  0.4222974 -0.5040606   -0.5441618         NA -0.4195023
## year          0.5814695 -0.3467172   -0.3698041         NA -0.3079004
## origin        0.5636979 -0.5649716   -0.6106643         NA -0.5812652
##              acceleration       year     origin
## mpg             0.4222974  0.5814695  0.5636979
## cylinders      -0.5040606 -0.3467172 -0.5649716
## displacement   -0.5441618 -0.3698041 -0.6106643
## horsepower             NA         NA         NA
## weight         -0.4195023 -0.3079004 -0.5812652
## acceleration    1.0000000  0.2829009  0.2100836
## year            0.2829009  1.0000000  0.1843141
## origin          0.2100836  0.1843141  1.0000000
\end{verbatim}

\emph{(c) Use the lm() function to perform a multiple linear regression
with mpg as the response and all other variables except name as the
predictors. Use the summary() function to print the results. Comment on
the output.}

\begin{Shaded}
\begin{Highlighting}[]
\NormalTok{mlmmpg }\OtherTok{=} \FunctionTok{lm}\NormalTok{(mpg}\SpecialCharTok{\textasciitilde{}}\NormalTok{cylinders }\SpecialCharTok{+}\NormalTok{ displacement }\SpecialCharTok{+}\NormalTok{ horsepower }\SpecialCharTok{+}\NormalTok{ weight }\SpecialCharTok{+}\NormalTok{ acceleration }\SpecialCharTok{+}\NormalTok{ year }\SpecialCharTok{+}\NormalTok{ origin, }\AttributeTok{data =}\NormalTok{ Auto)}
\FunctionTok{summary}\NormalTok{(mlmmpg)}
\end{Highlighting}
\end{Shaded}

\begin{verbatim}
## 
## Call:
## lm(formula = mpg ~ cylinders + displacement + horsepower + weight + 
##     acceleration + year + origin, data = Auto)
## 
## Residuals:
##     Min      1Q  Median      3Q     Max 
## -9.5903 -2.1565 -0.1169  1.8690 13.0604 
## 
## Coefficients:
##                Estimate Std. Error t value Pr(>|t|)    
## (Intercept)  -17.218435   4.644294  -3.707  0.00024 ***
## cylinders     -0.493376   0.323282  -1.526  0.12780    
## displacement   0.019896   0.007515   2.647  0.00844 ** 
## horsepower    -0.016951   0.013787  -1.230  0.21963    
## weight        -0.006474   0.000652  -9.929  < 2e-16 ***
## acceleration   0.080576   0.098845   0.815  0.41548    
## year           0.750773   0.050973  14.729  < 2e-16 ***
## origin         1.426141   0.278136   5.127 4.67e-07 ***
## ---
## Signif. codes:  0 '***' 0.001 '**' 0.01 '*' 0.05 '.' 0.1 ' ' 1
## 
## Residual standard error: 3.328 on 384 degrees of freedom
##   (5 observations deleted due to missingness)
## Multiple R-squared:  0.8215, Adjusted R-squared:  0.8182 
## F-statistic: 252.4 on 7 and 384 DF,  p-value: < 2.2e-16
\end{verbatim}

The weight, year, and origin of the car appear to be the most
statistically significant predictors, with p-values close to 0,
indicating a strong relationship with mpg. The displacement predictor
also seems statistically significant with a p-value close to 0.001. The
remaining predictors have p-values closer to 1, suggesting they lack
statistical significance in explaining variations in mpg.

The origin, year, and displacement predictors show a positive
relationship with mpg. Specifically, origin has the strongest effect on
mpg, with a coefficient of 1.4261 for one category relative to the
baseline, followed by year with an effect of 0.7508 for each unit
increase. Displacement has a smaller positive effect, with a coefficient
of 0.0199 per unit increase. Conversely, weight exhibits a negative
effect, reducing mpg by 0.0065 for every unit increase in weight.

\emph{(d) Use the plot() function to produce diagnostic plots of the
linear regression fit. Comment on any problems you see with the fit. Do
the residual plots suggest any unusually large outliers? Does the
leverage plot identify any observations with unusually high leverage?}

\begin{Shaded}
\begin{Highlighting}[]
\FunctionTok{par}\NormalTok{(}\AttributeTok{mfrow =} \FunctionTok{c}\NormalTok{(}\DecValTok{2}\NormalTok{,}\DecValTok{2}\NormalTok{)) }\CommentTok{\#Set plot layout to 2x2 grid}
\FunctionTok{plot}\NormalTok{(mlmmpg) }\CommentTok{\#Generate plot}
\end{Highlighting}
\end{Shaded}

\includegraphics{Econ573_ProblemSet2_Part2_files/figure-latex/unnamed-chunk-7-1.pdf}

The diagnostic plots reveal several insights about the model fit. The
Q-Q plot shows that the residuals are approximately normally
distributed, with minor deviations at the tails. The Residuals
vs.~Fitted plot suggests a slight non-linearity in the relationship
between the predictors and mpg. The Scale-Location plot indicates mild
heteroscedasticity, with residual variability increasing slightly as
fitted values increase. Finally, the Residuals vs.~Leverage plot does
not highlight any high-leverage or overly influential observations,
suggesting that no single data point disproportionately affects the
model.

\emph{(e) Use the} * \emph{and : symbols to fit linear regression models
with interaction effects. Do any interactions appear to be statistically
significant?}

We can run a linear regression model incorporating interaction terms
between the most statistically significant predictors to assess whether
these interactions provide additional explanatory power for
understanding the variability in miles per gallon (mpg).

\begin{Shaded}
\begin{Highlighting}[]
\NormalTok{mpg\_interac }\OtherTok{=} \FunctionTok{lm}\NormalTok{(mpg}\SpecialCharTok{\textasciitilde{}}\NormalTok{ cylinders }\SpecialCharTok{+}\NormalTok{ displacement }\SpecialCharTok{+}\NormalTok{ horsepower }\SpecialCharTok{+}\NormalTok{ weight }\SpecialCharTok{+}\NormalTok{ acceleration }\SpecialCharTok{+}\NormalTok{ year }\SpecialCharTok{+}\NormalTok{ origin }\SpecialCharTok{+}\NormalTok{ weight}\SpecialCharTok{*}\NormalTok{year }\SpecialCharTok{+}\NormalTok{ weight}\SpecialCharTok{*}\NormalTok{origin }\SpecialCharTok{+}\NormalTok{ weight}\SpecialCharTok{*}\NormalTok{displacement }\SpecialCharTok{+}\NormalTok{ year}\SpecialCharTok{*}\NormalTok{origin }\SpecialCharTok{+}\NormalTok{ year}\SpecialCharTok{*}\NormalTok{displacement }\SpecialCharTok{+}\NormalTok{ origin}\SpecialCharTok{*}\NormalTok{ displacement }\SpecialCharTok{+}\NormalTok{ horsepower}\SpecialCharTok{*}\NormalTok{year }\SpecialCharTok{+}\NormalTok{ horsepower}\SpecialCharTok{*}\NormalTok{origin }\SpecialCharTok{+}\NormalTok{ horsepower}\SpecialCharTok{*}\NormalTok{weight}\SpecialCharTok{+}\NormalTok{ horsepower}\SpecialCharTok{*}\NormalTok{displacement, }\AttributeTok{data =}\NormalTok{ Auto)}
\FunctionTok{summary}\NormalTok{(mpg\_interac)}
\end{Highlighting}
\end{Shaded}

\begin{verbatim}
## 
## Call:
## lm(formula = mpg ~ cylinders + displacement + horsepower + weight + 
##     acceleration + year + origin + weight * year + weight * origin + 
##     weight * displacement + year * origin + year * displacement + 
##     origin * displacement + horsepower * year + horsepower * 
##     origin + horsepower * weight + horsepower * displacement, 
##     data = Auto)
## 
## Residuals:
##     Min      1Q  Median      3Q     Max 
## -7.1616 -1.4868 -0.0497  1.3327 11.3031 
## 
## Coefficients:
##                           Estimate Std. Error t value Pr(>|t|)    
## (Intercept)             -5.093e+01  2.680e+01  -1.900  0.05818 .  
## cylinders                4.073e-01  3.108e-01   1.311  0.19080    
## displacement            -2.015e-01  1.203e-01  -1.676  0.09461 .  
## horsepower               9.445e-01  2.328e-01   4.058 6.03e-05 ***
## weight                  -1.644e-02  1.568e-02  -1.048  0.29523    
## acceleration            -7.872e-02  9.972e-02  -0.789  0.43036    
## year                     1.400e+00  3.320e-01   4.217 3.10e-05 ***
## origin                  -3.146e+00  5.495e+00  -0.573  0.56731    
## weight:year              7.611e-05  1.905e-04   0.400  0.68968    
## weight:origin            1.263e-03  1.131e-03   1.117  0.26474    
## displacement:weight      2.669e-05  6.345e-06   4.207 3.24e-05 ***
## year:origin              3.171e-02  6.966e-02   0.455  0.64919    
## displacement:year        1.123e-03  1.441e-03   0.779  0.43639    
## displacement:origin      2.687e-02  1.299e-02   2.068  0.03928 *  
## horsepower:year         -1.164e-02  2.893e-03  -4.024 6.93e-05 ***
## horsepower:origin       -5.776e-02  2.056e-02  -2.810  0.00522 ** 
## horsepower:weight       -1.419e-05  1.827e-05  -0.777  0.43778    
## displacement:horsepower -2.864e-05  1.405e-04  -0.204  0.83855    
## ---
## Signif. codes:  0 '***' 0.001 '**' 0.01 '*' 0.05 '.' 0.1 ' ' 1
## 
## Residual standard error: 2.755 on 374 degrees of freedom
##   (5 observations deleted due to missingness)
## Multiple R-squared:  0.8808, Adjusted R-squared:  0.8754 
## F-statistic: 162.6 on 17 and 374 DF,  p-value: < 2.2e-16
\end{verbatim}

From the regression results, we observe that most interaction
coefficients are not only small but also statistically insignificant.
However, there are a few exceptions. With a p-value close to 0, the
interaction terms weight:displacement and horsepower:year provide strong
evidence of statistical significance, with an effect of 1.814e−5 and
6.93e-05, respectively, on mpg. The interaction terms horsepower:origin
and displacement:origin also appear to be somewhat significant, with the
former having a p-value close to 0.001 and the latter close to 0.01.

Additionally, we observe that the inclusion of interaction terms causes
for the displacement, weight and origin predictors that were previously
determined to be statistically significant to become insignificant.
However, in the case of horsepower, strong evidence for its statistical
significance was now found. This effect can be attributed to increased
multicollinearity between the interaction terms and their associated
individual predictors.

\emph{(f) Try a few different transformations of the variables, such as
log(X), √ X, X2. Comment on your findings.}

Logarithmic Transformation of Weight:

\begin{Shaded}
\begin{Highlighting}[]
\NormalTok{log\_lmmpg }\OtherTok{=} \FunctionTok{lm}\NormalTok{(mpg }\SpecialCharTok{\textasciitilde{}}\NormalTok{ cylinders }\SpecialCharTok{+}\NormalTok{ displacement }\SpecialCharTok{+}\NormalTok{ horsepower }\SpecialCharTok{+} \FunctionTok{log}\NormalTok{(weight) }\SpecialCharTok{+}\NormalTok{ acceleration }\SpecialCharTok{+}\NormalTok{ year }\SpecialCharTok{+}\NormalTok{ origin, }\AttributeTok{data =}\NormalTok{ Auto)}
\FunctionTok{summary}\NormalTok{(log\_lmmpg)}
\end{Highlighting}
\end{Shaded}

\begin{verbatim}
## 
## Call:
## lm(formula = mpg ~ cylinders + displacement + horsepower + log(weight) + 
##     acceleration + year + origin, data = Auto)
## 
## Residuals:
##    Min     1Q Median     3Q    Max 
## -9.382 -1.973 -0.016  1.681 12.803 
## 
## Coefficients:
##                Estimate Std. Error t value Pr(>|t|)    
## (Intercept)  133.138912  11.531157  11.546  < 2e-16 ***
## cylinders     -0.432749   0.299977  -1.443 0.149946    
## displacement   0.020977   0.006825   3.074 0.002265 ** 
## horsepower    -0.010072   0.012546  -0.803 0.422593    
## log(weight)  -21.858785   1.651400 -13.237  < 2e-16 ***
## acceleration   0.135179   0.089798   1.505 0.133051    
## year           0.788784   0.047596  16.573  < 2e-16 ***
## origin         1.011407   0.262262   3.856 0.000135 ***
## ---
## Signif. codes:  0 '***' 0.001 '**' 0.01 '*' 0.05 '.' 0.1 ' ' 1
## 
## Residual standard error: 3.091 on 384 degrees of freedom
##   (5 observations deleted due to missingness)
## Multiple R-squared:  0.8459, Adjusted R-squared:  0.8431 
## F-statistic: 301.2 on 7 and 384 DF,  p-value: < 2.2e-16
\end{verbatim}

\begin{Shaded}
\begin{Highlighting}[]
\FunctionTok{par}\NormalTok{(}\AttributeTok{mfrow =} \FunctionTok{c}\NormalTok{(}\DecValTok{2}\NormalTok{,}\DecValTok{2}\NormalTok{))}
\FunctionTok{plot}\NormalTok{(log\_lmmpg)}
\end{Highlighting}
\end{Shaded}

\includegraphics{Econ573_ProblemSet2_Part2_files/figure-latex/unnamed-chunk-10-1.pdf}

With the logarithmic transformation of the weight predictor, we don't
see any changes in the significance of the predictors. However, we do
see an increase in the Adjusted R-squared compared to the original model
(0.8431 \textgreater{} 0.8182)indicating a better fit. Residual plots
exhibit virtually no changes compared to the original model.

Square Root Transformation of Acceleration and Horsepower:

\begin{Shaded}
\begin{Highlighting}[]
\NormalTok{sqrt\_lmmpg }\OtherTok{=} \FunctionTok{lm}\NormalTok{(mpg }\SpecialCharTok{\textasciitilde{}}\NormalTok{ cylinders }\SpecialCharTok{+}\NormalTok{ displacement }\SpecialCharTok{+} \FunctionTok{sqrt}\NormalTok{(horsepower) }\SpecialCharTok{+}\NormalTok{ weight }\SpecialCharTok{+} \FunctionTok{sqrt}\NormalTok{(acceleration) }\SpecialCharTok{+}\NormalTok{ year }\SpecialCharTok{+}\NormalTok{ origin, }\AttributeTok{data =}\NormalTok{ Auto)}
\FunctionTok{summary}\NormalTok{(sqrt\_lmmpg)}
\end{Highlighting}
\end{Shaded}

\begin{verbatim}
## 
## Call:
## lm(formula = mpg ~ cylinders + displacement + sqrt(horsepower) + 
##     weight + sqrt(acceleration) + year + origin, data = Auto)
## 
## Residuals:
##    Min     1Q Median     3Q    Max 
## -9.615 -2.038 -0.138  1.761 12.945 
## 
## Coefficients:
##                      Estimate Std. Error t value Pr(>|t|)    
## (Intercept)        -1.9378664  6.6729385  -0.290  0.77166    
## cylinders          -0.5175082  0.3161160  -1.637  0.10243    
## displacement        0.0210864  0.0072379   2.913  0.00379 ** 
## sqrt(horsepower)   -1.2539322  0.3141308  -3.992 7.86e-05 ***
## weight             -0.0052428  0.0006955  -7.539 3.46e-13 ***
## sqrt(acceleration) -1.2478493  0.8448937  -1.477  0.14051    
## year                0.7220943  0.0500773  14.420  < 2e-16 ***
## origin              1.5109369  0.2699801   5.596 4.17e-08 ***
## ---
## Signif. codes:  0 '***' 0.001 '**' 0.01 '*' 0.05 '.' 0.1 ' ' 1
## 
## Residual standard error: 3.272 on 384 degrees of freedom
##   (5 observations deleted due to missingness)
## Multiple R-squared:  0.8274, Adjusted R-squared:  0.8243 
## F-statistic:   263 on 7 and 384 DF,  p-value: < 2.2e-16
\end{verbatim}

\begin{Shaded}
\begin{Highlighting}[]
\FunctionTok{par}\NormalTok{(}\AttributeTok{mfrow =} \FunctionTok{c}\NormalTok{(}\DecValTok{2}\NormalTok{,}\DecValTok{2}\NormalTok{))}
\FunctionTok{plot}\NormalTok{(sqrt\_lmmpg)}
\end{Highlighting}
\end{Shaded}

\includegraphics{Econ573_ProblemSet2_Part2_files/figure-latex/unnamed-chunk-12-1.pdf}

With the square root transformation applied to acceleration and
horsepower, we observe that the transformed horsepower predictor gained
statistical significance. Additionally, the Adjusted R-squared value
increased compared to the original model (0.8243 \textgreater{} 0.8182).
However, residual plots remain virtually unchanged from those of the
original model.

Quadratic Transformation of Horsepower and Weight:

\begin{Shaded}
\begin{Highlighting}[]
\NormalTok{quad\_lmmpg }\OtherTok{=} \FunctionTok{lm}\NormalTok{(mpg }\SpecialCharTok{\textasciitilde{}}\NormalTok{ cylinders }\SpecialCharTok{+}\NormalTok{ displacement }\SpecialCharTok{+} \FunctionTok{I}\NormalTok{(horsepower}\SpecialCharTok{\^{}}\DecValTok{2}\NormalTok{) }\SpecialCharTok{+} \FunctionTok{I}\NormalTok{(weight}\SpecialCharTok{\^{}}\DecValTok{2}\NormalTok{) }\SpecialCharTok{+}\NormalTok{ acceleration }\SpecialCharTok{+}\NormalTok{ year }\SpecialCharTok{+}\NormalTok{ origin, }\AttributeTok{data =}\NormalTok{ Auto)}
\FunctionTok{summary}\NormalTok{(quad\_lmmpg)}
\end{Highlighting}
\end{Shaded}

\begin{verbatim}
## 
## Call:
## lm(formula = mpg ~ cylinders + displacement + I(horsepower^2) + 
##     I(weight^2) + acceleration + year + origin, data = Auto)
## 
## Residuals:
##      Min       1Q   Median       3Q      Max 
## -10.1323  -2.3350  -0.0783   2.0358  13.4260 
## 
## Coefficients:
##                   Estimate Std. Error t value Pr(>|t|)    
## (Intercept)     -2.817e+01  4.802e+00  -5.867 9.61e-09 ***
## cylinders       -4.925e-01  3.565e-01  -1.381    0.168    
## displacement    -3.060e-03  8.447e-03  -0.362    0.717    
## I(horsepower^2)  7.734e-05  4.744e-05   1.630    0.104    
## I(weight^2)     -7.649e-07  9.869e-08  -7.750 8.29e-14 ***
## acceleration     1.350e-01  9.512e-02   1.420    0.157    
## year             7.491e-01  5.483e-02  13.661  < 2e-16 ***
## origin           1.438e+00  3.053e-01   4.709 3.48e-06 ***
## ---
## Signif. codes:  0 '***' 0.001 '**' 0.01 '*' 0.05 '.' 0.1 ' ' 1
## 
## Residual standard error: 3.613 on 384 degrees of freedom
##   (5 observations deleted due to missingness)
## Multiple R-squared:  0.7896, Adjusted R-squared:  0.7858 
## F-statistic: 205.9 on 7 and 384 DF,  p-value: < 2.2e-16
\end{verbatim}

\begin{Shaded}
\begin{Highlighting}[]
\FunctionTok{par}\NormalTok{(}\AttributeTok{mfrow =} \FunctionTok{c}\NormalTok{(}\DecValTok{2}\NormalTok{,}\DecValTok{2}\NormalTok{))}
\FunctionTok{plot}\NormalTok{(quad\_lmmpg)}
\end{Highlighting}
\end{Shaded}

\includegraphics{Econ573_ProblemSet2_Part2_files/figure-latex/unnamed-chunk-14-1.pdf}

When applying a quadratic transformation to horsepower and weight, the
acceleration predictor lost its statistical significance. Furthermore,
the transformed model exhibited a lower Adjusted R-squared value
compared to the original model (0.7858 \textless{} 0.8182) suggesting a
worse overall fit. Residual plots also show no noticeable differences
from the original model.

\emph{Question 13}

\begin{Shaded}
\begin{Highlighting}[]
\FunctionTok{set.seed}\NormalTok{(}\DecValTok{1}\NormalTok{)}
\end{Highlighting}
\end{Shaded}

\emph{(a) Using the rnorm() function, create a vector, x, containing 100
observations drawn from a N(0, 1) distribution. This represents a
feature, X. }

\begin{Shaded}
\begin{Highlighting}[]
\NormalTok{x }\OtherTok{=} \FunctionTok{rnorm}\NormalTok{(}\DecValTok{100}\NormalTok{, }\AttributeTok{mean =} \DecValTok{0}\NormalTok{, }\AttributeTok{sd =} \DecValTok{1}\NormalTok{)}
\end{Highlighting}
\end{Shaded}

\emph{(b) Using the rnorm() function, create a vector, eps, containing
100 observations drawn from a N(0, 0.25) distribution---a normal
distribution with mean zero and variance 0.25. }

\begin{Shaded}
\begin{Highlighting}[]
\NormalTok{eps }\OtherTok{=} \FunctionTok{rnorm}\NormalTok{(}\DecValTok{100}\NormalTok{, }\AttributeTok{mean =} \DecValTok{0}\NormalTok{, }\AttributeTok{sd =} \FunctionTok{sqrt}\NormalTok{(}\FloatTok{0.25}\NormalTok{))}
\end{Highlighting}
\end{Shaded}

\emph{(c) Using x and eps, generate a vector y according to the model Y
= −1+0.5X + ϵ. (3.39) What is the length of the vector y? What are the
values of β0 and β1 in this linear model?}

\begin{Shaded}
\begin{Highlighting}[]
\NormalTok{y }\OtherTok{=} \SpecialCharTok{{-}}\DecValTok{1} \SpecialCharTok{+} \FloatTok{0.5}\SpecialCharTok{*}\NormalTok{x }\SpecialCharTok{+}\NormalTok{ eps}
\end{Highlighting}
\end{Shaded}

The length of the vector will be 100, the same as both x and eps. The
value of the intercept parameter is \(\beta_0 = -1\), while the value of
the slope parameter is \(\beta_1 = 0.5\).

\emph{(d) Create a scatterplot displaying the relationship between x and
y. Comment on what you observe.}

\begin{Shaded}
\begin{Highlighting}[]
\FunctionTok{plot}\NormalTok{(x, y, }\AttributeTok{main=}\StringTok{"Scatterplot of X vs Y"}\NormalTok{, }\AttributeTok{xlab=}\StringTok{"X"}\NormalTok{, }\AttributeTok{ylab=}\StringTok{"Y"}\NormalTok{, }\AttributeTok{pch=}\DecValTok{19}\NormalTok{, }\AttributeTok{col=}\StringTok{"cadetblue1"}\NormalTok{)}
\FunctionTok{abline}\NormalTok{(}\SpecialCharTok{{-}}\DecValTok{1}\NormalTok{, }\FloatTok{0.5}\NormalTok{, }\AttributeTok{col=}\StringTok{"black"}\NormalTok{, }\AttributeTok{lwd=}\DecValTok{2}\NormalTok{, }\AttributeTok{lty=}\DecValTok{2}\NormalTok{)}
\end{Highlighting}
\end{Shaded}

\includegraphics{Econ573_ProblemSet2_Part2_files/figure-latex/unnamed-chunk-19-1.pdf}
The scatterplot shows a positive, approximately linear relationship
between X and Y. As X increases, Y tends to increase as well. However,
the random noise introduced by the eps term creates some variability
around the linear trend, resulting in a moderate spread of points across
the regression line.

\emph{(e) Fit a least squares linear model to predict y using x. Comment
on the model obtained. How do βˆ0 and βˆ1 compare to β0 and β1?}

\begin{Shaded}
\begin{Highlighting}[]
\NormalTok{lm }\OtherTok{=} \FunctionTok{lm}\NormalTok{(y}\SpecialCharTok{\textasciitilde{}}\NormalTok{x)}
\FunctionTok{summary}\NormalTok{(lm)}
\end{Highlighting}
\end{Shaded}

\begin{verbatim}
## 
## Call:
## lm(formula = y ~ x)
## 
## Residuals:
##      Min       1Q   Median       3Q      Max 
## -0.93842 -0.30688 -0.06975  0.26970  1.17309 
## 
## Coefficients:
##             Estimate Std. Error t value Pr(>|t|)    
## (Intercept) -1.01885    0.04849 -21.010  < 2e-16 ***
## x            0.49947    0.05386   9.273 4.58e-15 ***
## ---
## Signif. codes:  0 '***' 0.001 '**' 0.01 '*' 0.05 '.' 0.1 ' ' 1
## 
## Residual standard error: 0.4814 on 98 degrees of freedom
## Multiple R-squared:  0.4674, Adjusted R-squared:  0.4619 
## F-statistic: 85.99 on 1 and 98 DF,  p-value: 4.583e-15
\end{verbatim}

The estimated intercept \(\hat{\beta}_0 = -0.98632\) is close to the
true value \(\beta_0 = -1\). Similarly, the estimated coefficient for x
\(\hat{\beta}_1 = 0.51058\) is close to the true value
\(\beta_1 = 0.5\). The small deviation of these estimated values from
the true values is due to the random noise introduced by the error term,
ϵ. However, we can conclude that the model accurately captures the true
relationship between x and y.

Additionally, there is strong evidence that x is a statistically
significant predictor, as indicated by a p-value of \(2*10^{-16}\).
However, the R-squared value for the model is very low at 0.4346, or the
model is relatively low at 0.4346, meaning that only 43.46\% of the
variability in Y is explained by the predictor variable, x, in this
linear regression model.

\emph{Question 15}

\begin{Shaded}
\begin{Highlighting}[]
\NormalTok{Boston }\OtherTok{\textless{}{-}} \FunctionTok{read.csv}\NormalTok{(}\StringTok{"C:/Users/mateo/OneDrive {-} University of North Carolina at Chapel Hill/Courses/Spring 2025/Econ 573/ALL+CSV+FILES+{-}+2nd+Edition+{-}+corrected/ALL CSV FILES {-} 2nd Edition/Boston.csv"}\NormalTok{, }\AttributeTok{header =} \ConstantTok{TRUE}\NormalTok{)}
\NormalTok{Boston }\OtherTok{\textless{}{-}}\NormalTok{ Boston[, }\FunctionTok{sapply}\NormalTok{(Boston, is.numeric)]}
\NormalTok{Boston }\OtherTok{\textless{}{-}}\NormalTok{ Boston[, }\FunctionTok{sapply}\NormalTok{(Boston, }\ControlFlowTok{function}\NormalTok{(col) }\FunctionTok{length}\NormalTok{(}\FunctionTok{unique}\NormalTok{(col)) }\SpecialCharTok{\textgreater{}} \DecValTok{1}\NormalTok{)]}
\NormalTok{Boston }\OtherTok{\textless{}{-}}\NormalTok{ Boston[, }\SpecialCharTok{!}\FunctionTok{names}\NormalTok{(Boston) }\SpecialCharTok{\%in\%} \StringTok{"X"}\NormalTok{]}
\NormalTok{Boston}\SpecialCharTok{$}\NormalTok{chas }\OtherTok{\textless{}{-}} \FunctionTok{as.factor}\NormalTok{(Boston}\SpecialCharTok{$}\NormalTok{chas)}
\end{Highlighting}
\end{Shaded}

\emph{(a) For each predictor, fit a simple linear regression model to
predict the response. Describe your results. In which of the models is
there a statistically significant association between the predictor and
the response? Create some plots to back up your assertions.}

\begin{Shaded}
\begin{Highlighting}[]
\NormalTok{model\_zn }\OtherTok{=} \FunctionTok{lm}\NormalTok{(crim }\SpecialCharTok{\textasciitilde{}}\NormalTok{ zn, }\AttributeTok{data =}\NormalTok{ Boston)}
\NormalTok{model\_chas }\OtherTok{=} \FunctionTok{lm}\NormalTok{(crim }\SpecialCharTok{\textasciitilde{}}\NormalTok{ chas, }\AttributeTok{data =}\NormalTok{ Boston)}
\NormalTok{model\_indus }\OtherTok{=} \FunctionTok{lm}\NormalTok{(crim }\SpecialCharTok{\textasciitilde{}}\NormalTok{ indus, }\AttributeTok{data =}\NormalTok{ Boston)}
\NormalTok{model\_nox }\OtherTok{=} \FunctionTok{lm}\NormalTok{(crim }\SpecialCharTok{\textasciitilde{}}\NormalTok{ nox, }\AttributeTok{data =}\NormalTok{ Boston)}
\NormalTok{model\_rm }\OtherTok{=} \FunctionTok{lm}\NormalTok{(crim }\SpecialCharTok{\textasciitilde{}}\NormalTok{ rm, }\AttributeTok{data =}\NormalTok{ Boston)}
\NormalTok{model\_age }\OtherTok{=} \FunctionTok{lm}\NormalTok{(crim }\SpecialCharTok{\textasciitilde{}}\NormalTok{ age, }\AttributeTok{data =}\NormalTok{ Boston)}
\NormalTok{model\_dis }\OtherTok{=} \FunctionTok{lm}\NormalTok{(crim }\SpecialCharTok{\textasciitilde{}}\NormalTok{ dis, }\AttributeTok{data =}\NormalTok{ Boston)}
\NormalTok{model\_rad }\OtherTok{=} \FunctionTok{lm}\NormalTok{(crim }\SpecialCharTok{\textasciitilde{}}\NormalTok{ rad, }\AttributeTok{data =}\NormalTok{ Boston)}
\NormalTok{model\_tax }\OtherTok{=} \FunctionTok{lm}\NormalTok{(crim }\SpecialCharTok{\textasciitilde{}}\NormalTok{ tax, }\AttributeTok{data =}\NormalTok{ Boston)}
\NormalTok{model\_ptratio }\OtherTok{=} \FunctionTok{lm}\NormalTok{(crim }\SpecialCharTok{\textasciitilde{}}\NormalTok{ ptratio, }\AttributeTok{data =}\NormalTok{ Boston)}
\NormalTok{model\_lstat }\OtherTok{=} \FunctionTok{lm}\NormalTok{(crim }\SpecialCharTok{\textasciitilde{}}\NormalTok{ lstat, }\AttributeTok{data =}\NormalTok{ Boston)}
\NormalTok{model\_medv }\OtherTok{=} \FunctionTok{lm}\NormalTok{(crim }\SpecialCharTok{\textasciitilde{}}\NormalTok{ medv, }\AttributeTok{data =}\NormalTok{ Boston)}

\FunctionTok{summary}\NormalTok{(model\_zn)}
\end{Highlighting}
\end{Shaded}

\begin{verbatim}
## 
## Call:
## lm(formula = crim ~ zn, data = Boston)
## 
## Residuals:
##    Min     1Q Median     3Q    Max 
## -4.429 -4.222 -2.620  1.250 84.523 
## 
## Coefficients:
##             Estimate Std. Error t value Pr(>|t|)    
## (Intercept)  4.45369    0.41722  10.675  < 2e-16 ***
## zn          -0.07393    0.01609  -4.594 5.51e-06 ***
## ---
## Signif. codes:  0 '***' 0.001 '**' 0.01 '*' 0.05 '.' 0.1 ' ' 1
## 
## Residual standard error: 8.435 on 504 degrees of freedom
## Multiple R-squared:  0.04019,    Adjusted R-squared:  0.03828 
## F-statistic:  21.1 on 1 and 504 DF,  p-value: 5.506e-06
\end{verbatim}

\begin{Shaded}
\begin{Highlighting}[]
\FunctionTok{summary}\NormalTok{(model\_chas)}
\end{Highlighting}
\end{Shaded}

\begin{verbatim}
## 
## Call:
## lm(formula = crim ~ chas, data = Boston)
## 
## Residuals:
##    Min     1Q Median     3Q    Max 
## -3.738 -3.661 -3.435  0.018 85.232 
## 
## Coefficients:
##             Estimate Std. Error t value Pr(>|t|)    
## (Intercept)   3.7444     0.3961   9.453   <2e-16 ***
## chas1        -1.8928     1.5061  -1.257    0.209    
## ---
## Signif. codes:  0 '***' 0.001 '**' 0.01 '*' 0.05 '.' 0.1 ' ' 1
## 
## Residual standard error: 8.597 on 504 degrees of freedom
## Multiple R-squared:  0.003124,   Adjusted R-squared:  0.001146 
## F-statistic: 1.579 on 1 and 504 DF,  p-value: 0.2094
\end{verbatim}

\begin{Shaded}
\begin{Highlighting}[]
\FunctionTok{summary}\NormalTok{(model\_indus)}
\end{Highlighting}
\end{Shaded}

\begin{verbatim}
## 
## Call:
## lm(formula = crim ~ indus, data = Boston)
## 
## Residuals:
##     Min      1Q  Median      3Q     Max 
## -11.972  -2.698  -0.736   0.712  81.813 
## 
## Coefficients:
##             Estimate Std. Error t value Pr(>|t|)    
## (Intercept) -2.06374    0.66723  -3.093  0.00209 ** 
## indus        0.50978    0.05102   9.991  < 2e-16 ***
## ---
## Signif. codes:  0 '***' 0.001 '**' 0.01 '*' 0.05 '.' 0.1 ' ' 1
## 
## Residual standard error: 7.866 on 504 degrees of freedom
## Multiple R-squared:  0.1653, Adjusted R-squared:  0.1637 
## F-statistic: 99.82 on 1 and 504 DF,  p-value: < 2.2e-16
\end{verbatim}

\begin{Shaded}
\begin{Highlighting}[]
\FunctionTok{summary}\NormalTok{(model\_nox)}
\end{Highlighting}
\end{Shaded}

\begin{verbatim}
## 
## Call:
## lm(formula = crim ~ nox, data = Boston)
## 
## Residuals:
##     Min      1Q  Median      3Q     Max 
## -12.371  -2.738  -0.974   0.559  81.728 
## 
## Coefficients:
##             Estimate Std. Error t value Pr(>|t|)    
## (Intercept)  -13.720      1.699  -8.073 5.08e-15 ***
## nox           31.249      2.999  10.419  < 2e-16 ***
## ---
## Signif. codes:  0 '***' 0.001 '**' 0.01 '*' 0.05 '.' 0.1 ' ' 1
## 
## Residual standard error: 7.81 on 504 degrees of freedom
## Multiple R-squared:  0.1772, Adjusted R-squared:  0.1756 
## F-statistic: 108.6 on 1 and 504 DF,  p-value: < 2.2e-16
\end{verbatim}

\begin{Shaded}
\begin{Highlighting}[]
\FunctionTok{summary}\NormalTok{(model\_rm)}
\end{Highlighting}
\end{Shaded}

\begin{verbatim}
## 
## Call:
## lm(formula = crim ~ rm, data = Boston)
## 
## Residuals:
##    Min     1Q Median     3Q    Max 
## -6.604 -3.952 -2.654  0.989 87.197 
## 
## Coefficients:
##             Estimate Std. Error t value Pr(>|t|)    
## (Intercept)   20.482      3.365   6.088 2.27e-09 ***
## rm            -2.684      0.532  -5.045 6.35e-07 ***
## ---
## Signif. codes:  0 '***' 0.001 '**' 0.01 '*' 0.05 '.' 0.1 ' ' 1
## 
## Residual standard error: 8.401 on 504 degrees of freedom
## Multiple R-squared:  0.04807,    Adjusted R-squared:  0.04618 
## F-statistic: 25.45 on 1 and 504 DF,  p-value: 6.347e-07
\end{verbatim}

\begin{Shaded}
\begin{Highlighting}[]
\FunctionTok{summary}\NormalTok{(model\_age)}
\end{Highlighting}
\end{Shaded}

\begin{verbatim}
## 
## Call:
## lm(formula = crim ~ age, data = Boston)
## 
## Residuals:
##    Min     1Q Median     3Q    Max 
## -6.789 -4.257 -1.230  1.527 82.849 
## 
## Coefficients:
##             Estimate Std. Error t value Pr(>|t|)    
## (Intercept) -3.77791    0.94398  -4.002 7.22e-05 ***
## age          0.10779    0.01274   8.463 2.85e-16 ***
## ---
## Signif. codes:  0 '***' 0.001 '**' 0.01 '*' 0.05 '.' 0.1 ' ' 1
## 
## Residual standard error: 8.057 on 504 degrees of freedom
## Multiple R-squared:  0.1244, Adjusted R-squared:  0.1227 
## F-statistic: 71.62 on 1 and 504 DF,  p-value: 2.855e-16
\end{verbatim}

\begin{Shaded}
\begin{Highlighting}[]
\FunctionTok{summary}\NormalTok{(model\_dis)}
\end{Highlighting}
\end{Shaded}

\begin{verbatim}
## 
## Call:
## lm(formula = crim ~ dis, data = Boston)
## 
## Residuals:
##    Min     1Q Median     3Q    Max 
## -6.708 -4.134 -1.527  1.516 81.674 
## 
## Coefficients:
##             Estimate Std. Error t value Pr(>|t|)    
## (Intercept)   9.4993     0.7304  13.006   <2e-16 ***
## dis          -1.5509     0.1683  -9.213   <2e-16 ***
## ---
## Signif. codes:  0 '***' 0.001 '**' 0.01 '*' 0.05 '.' 0.1 ' ' 1
## 
## Residual standard error: 7.965 on 504 degrees of freedom
## Multiple R-squared:  0.1441, Adjusted R-squared:  0.1425 
## F-statistic: 84.89 on 1 and 504 DF,  p-value: < 2.2e-16
\end{verbatim}

\begin{Shaded}
\begin{Highlighting}[]
\FunctionTok{summary}\NormalTok{(model\_rad)}
\end{Highlighting}
\end{Shaded}

\begin{verbatim}
## 
## Call:
## lm(formula = crim ~ rad, data = Boston)
## 
## Residuals:
##     Min      1Q  Median      3Q     Max 
## -10.164  -1.381  -0.141   0.660  76.433 
## 
## Coefficients:
##             Estimate Std. Error t value Pr(>|t|)    
## (Intercept) -2.28716    0.44348  -5.157 3.61e-07 ***
## rad          0.61791    0.03433  17.998  < 2e-16 ***
## ---
## Signif. codes:  0 '***' 0.001 '**' 0.01 '*' 0.05 '.' 0.1 ' ' 1
## 
## Residual standard error: 6.718 on 504 degrees of freedom
## Multiple R-squared:  0.3913, Adjusted R-squared:   0.39 
## F-statistic: 323.9 on 1 and 504 DF,  p-value: < 2.2e-16
\end{verbatim}

\begin{Shaded}
\begin{Highlighting}[]
\FunctionTok{summary}\NormalTok{(model\_tax)}
\end{Highlighting}
\end{Shaded}

\begin{verbatim}
## 
## Call:
## lm(formula = crim ~ tax, data = Boston)
## 
## Residuals:
##     Min      1Q  Median      3Q     Max 
## -12.513  -2.738  -0.194   1.065  77.696 
## 
## Coefficients:
##              Estimate Std. Error t value Pr(>|t|)    
## (Intercept) -8.528369   0.815809  -10.45   <2e-16 ***
## tax          0.029742   0.001847   16.10   <2e-16 ***
## ---
## Signif. codes:  0 '***' 0.001 '**' 0.01 '*' 0.05 '.' 0.1 ' ' 1
## 
## Residual standard error: 6.997 on 504 degrees of freedom
## Multiple R-squared:  0.3396, Adjusted R-squared:  0.3383 
## F-statistic: 259.2 on 1 and 504 DF,  p-value: < 2.2e-16
\end{verbatim}

\begin{Shaded}
\begin{Highlighting}[]
\FunctionTok{summary}\NormalTok{(model\_ptratio)}
\end{Highlighting}
\end{Shaded}

\begin{verbatim}
## 
## Call:
## lm(formula = crim ~ ptratio, data = Boston)
## 
## Residuals:
##    Min     1Q Median     3Q    Max 
## -7.654 -3.985 -1.912  1.825 83.353 
## 
## Coefficients:
##             Estimate Std. Error t value Pr(>|t|)    
## (Intercept) -17.6469     3.1473  -5.607 3.40e-08 ***
## ptratio       1.1520     0.1694   6.801 2.94e-11 ***
## ---
## Signif. codes:  0 '***' 0.001 '**' 0.01 '*' 0.05 '.' 0.1 ' ' 1
## 
## Residual standard error: 8.24 on 504 degrees of freedom
## Multiple R-squared:  0.08407,    Adjusted R-squared:  0.08225 
## F-statistic: 46.26 on 1 and 504 DF,  p-value: 2.943e-11
\end{verbatim}

\begin{Shaded}
\begin{Highlighting}[]
\FunctionTok{summary}\NormalTok{(model\_lstat)}
\end{Highlighting}
\end{Shaded}

\begin{verbatim}
## 
## Call:
## lm(formula = crim ~ lstat, data = Boston)
## 
## Residuals:
##     Min      1Q  Median      3Q     Max 
## -13.925  -2.822  -0.664   1.079  82.862 
## 
## Coefficients:
##             Estimate Std. Error t value Pr(>|t|)    
## (Intercept) -3.33054    0.69376  -4.801 2.09e-06 ***
## lstat        0.54880    0.04776  11.491  < 2e-16 ***
## ---
## Signif. codes:  0 '***' 0.001 '**' 0.01 '*' 0.05 '.' 0.1 ' ' 1
## 
## Residual standard error: 7.664 on 504 degrees of freedom
## Multiple R-squared:  0.2076, Adjusted R-squared:  0.206 
## F-statistic:   132 on 1 and 504 DF,  p-value: < 2.2e-16
\end{verbatim}

\begin{Shaded}
\begin{Highlighting}[]
\FunctionTok{summary}\NormalTok{(model\_medv)}
\end{Highlighting}
\end{Shaded}

\begin{verbatim}
## 
## Call:
## lm(formula = crim ~ medv, data = Boston)
## 
## Residuals:
##    Min     1Q Median     3Q    Max 
## -9.071 -4.022 -2.343  1.298 80.957 
## 
## Coefficients:
##             Estimate Std. Error t value Pr(>|t|)    
## (Intercept) 11.79654    0.93419   12.63   <2e-16 ***
## medv        -0.36316    0.03839   -9.46   <2e-16 ***
## ---
## Signif. codes:  0 '***' 0.001 '**' 0.01 '*' 0.05 '.' 0.1 ' ' 1
## 
## Residual standard error: 7.934 on 504 degrees of freedom
## Multiple R-squared:  0.1508, Adjusted R-squared:  0.1491 
## F-statistic: 89.49 on 1 and 504 DF,  p-value: < 2.2e-16
\end{verbatim}

\begin{Shaded}
\begin{Highlighting}[]
\FunctionTok{par}\NormalTok{(}\AttributeTok{mfrow =} \FunctionTok{c}\NormalTok{(}\DecValTok{3}\NormalTok{, }\DecValTok{4}\NormalTok{))}

\FunctionTok{plot}\NormalTok{(Boston}\SpecialCharTok{$}\NormalTok{zn, Boston}\SpecialCharTok{$}\NormalTok{crim, }\AttributeTok{main =} \StringTok{"zn vs crim"}\NormalTok{, }\AttributeTok{xlab =} \StringTok{"zn"}\NormalTok{, }\AttributeTok{ylab =} \StringTok{"Crime Rate"}\NormalTok{)}
\FunctionTok{abline}\NormalTok{(model\_zn, }\AttributeTok{col =} \StringTok{"red"}\NormalTok{, }\AttributeTok{lwd =} \DecValTok{2}\NormalTok{)}

\FunctionTok{plot}\NormalTok{(Boston}\SpecialCharTok{$}\NormalTok{indus, Boston}\SpecialCharTok{$}\NormalTok{crim, }\AttributeTok{main =} \StringTok{"indus vs crim"}\NormalTok{, }\AttributeTok{xlab =} \StringTok{"indus"}\NormalTok{, }\AttributeTok{ylab =} \StringTok{"Crime Rate"}\NormalTok{)}
\FunctionTok{abline}\NormalTok{(model\_indus, }\AttributeTok{col =} \StringTok{"red"}\NormalTok{, }\AttributeTok{lwd =} \DecValTok{2}\NormalTok{)}

\FunctionTok{plot}\NormalTok{(Boston}\SpecialCharTok{$}\NormalTok{nox, Boston}\SpecialCharTok{$}\NormalTok{crim, }\AttributeTok{main =} \StringTok{"nox vs crim"}\NormalTok{, }\AttributeTok{xlab =} \StringTok{"nox"}\NormalTok{, }\AttributeTok{ylab =} \StringTok{"Crime Rate"}\NormalTok{)}
\FunctionTok{abline}\NormalTok{(model\_nox, }\AttributeTok{col =} \StringTok{"red"}\NormalTok{, }\AttributeTok{lwd =} \DecValTok{2}\NormalTok{)}

\FunctionTok{plot}\NormalTok{(Boston}\SpecialCharTok{$}\NormalTok{rm, Boston}\SpecialCharTok{$}\NormalTok{crim, }\AttributeTok{main =} \StringTok{"rm vs crim"}\NormalTok{, }\AttributeTok{xlab =} \StringTok{"rm"}\NormalTok{, }\AttributeTok{ylab =} \StringTok{"Crime Rate"}\NormalTok{)}
\FunctionTok{abline}\NormalTok{(model\_rm, }\AttributeTok{col =} \StringTok{"red"}\NormalTok{, }\AttributeTok{lwd =} \DecValTok{2}\NormalTok{)}

\FunctionTok{plot}\NormalTok{(Boston}\SpecialCharTok{$}\NormalTok{age, Boston}\SpecialCharTok{$}\NormalTok{crim, }\AttributeTok{main =} \StringTok{"age vs crim"}\NormalTok{, }\AttributeTok{xlab =} \StringTok{"age"}\NormalTok{, }\AttributeTok{ylab =} \StringTok{"Crime Rate"}\NormalTok{)}
\FunctionTok{abline}\NormalTok{(model\_age, }\AttributeTok{col =} \StringTok{"red"}\NormalTok{, }\AttributeTok{lwd =} \DecValTok{2}\NormalTok{)}

\FunctionTok{plot}\NormalTok{(Boston}\SpecialCharTok{$}\NormalTok{dis, Boston}\SpecialCharTok{$}\NormalTok{crim, }\AttributeTok{main =} \StringTok{"dis vs crim"}\NormalTok{, }\AttributeTok{xlab =} \StringTok{"dis"}\NormalTok{, }\AttributeTok{ylab =} \StringTok{"Crime Rate"}\NormalTok{)}
\FunctionTok{abline}\NormalTok{(model\_dis, }\AttributeTok{col =} \StringTok{"red"}\NormalTok{, }\AttributeTok{lwd =} \DecValTok{2}\NormalTok{)}

\FunctionTok{plot}\NormalTok{(Boston}\SpecialCharTok{$}\NormalTok{rad, Boston}\SpecialCharTok{$}\NormalTok{crim, }\AttributeTok{main =} \StringTok{"rad vs crim"}\NormalTok{, }\AttributeTok{xlab =} \StringTok{"rad"}\NormalTok{, }\AttributeTok{ylab =} \StringTok{"Crime Rate"}\NormalTok{)}
\FunctionTok{abline}\NormalTok{(model\_rad, }\AttributeTok{col =} \StringTok{"red"}\NormalTok{, }\AttributeTok{lwd =} \DecValTok{2}\NormalTok{)}

\FunctionTok{plot}\NormalTok{(Boston}\SpecialCharTok{$}\NormalTok{tax, Boston}\SpecialCharTok{$}\NormalTok{crim, }\AttributeTok{main =} \StringTok{"tax vs crim"}\NormalTok{, }\AttributeTok{xlab =} \StringTok{"tax"}\NormalTok{, }\AttributeTok{ylab =} \StringTok{"Crime Rate"}\NormalTok{)}
\FunctionTok{abline}\NormalTok{(model\_tax, }\AttributeTok{col =} \StringTok{"red"}\NormalTok{, }\AttributeTok{lwd =} \DecValTok{2}\NormalTok{)}

\FunctionTok{plot}\NormalTok{(Boston}\SpecialCharTok{$}\NormalTok{ptratio, Boston}\SpecialCharTok{$}\NormalTok{crim, }\AttributeTok{main =} \StringTok{"ptratio vs crim"}\NormalTok{, }\AttributeTok{xlab =} \StringTok{"ptratio"}\NormalTok{, }\AttributeTok{ylab =} \StringTok{"Crime Rate"}\NormalTok{)}
\FunctionTok{abline}\NormalTok{(model\_ptratio, }\AttributeTok{col =} \StringTok{"red"}\NormalTok{, }\AttributeTok{lwd =} \DecValTok{2}\NormalTok{)}

\FunctionTok{plot}\NormalTok{(Boston}\SpecialCharTok{$}\NormalTok{lstat, Boston}\SpecialCharTok{$}\NormalTok{crim, }\AttributeTok{main =} \StringTok{"lstat vs crim"}\NormalTok{, }\AttributeTok{xlab =} \StringTok{"lstat"}\NormalTok{, }\AttributeTok{ylab =} \StringTok{"Crime Rate"}\NormalTok{)}
\FunctionTok{abline}\NormalTok{(model\_lstat, }\AttributeTok{col =} \StringTok{"red"}\NormalTok{, }\AttributeTok{lwd =} \DecValTok{2}\NormalTok{)}

\FunctionTok{plot}\NormalTok{(Boston}\SpecialCharTok{$}\NormalTok{medv, Boston}\SpecialCharTok{$}\NormalTok{crim, }\AttributeTok{main =} \StringTok{"medv vs crim"}\NormalTok{, }\AttributeTok{xlab =} \StringTok{"medv"}\NormalTok{, }\AttributeTok{ylab =} \StringTok{"Crime Rate"}\NormalTok{)}
\FunctionTok{abline}\NormalTok{(model\_medv, }\AttributeTok{col =} \StringTok{"red"}\NormalTok{, }\AttributeTok{lwd =} \DecValTok{2}\NormalTok{)}

\FunctionTok{par}\NormalTok{(}\AttributeTok{mfrow =} \FunctionTok{c}\NormalTok{(}\DecValTok{1}\NormalTok{, }\DecValTok{1}\NormalTok{))}
\end{Highlighting}
\end{Shaded}

\includegraphics{Econ573_ProblemSet2_Part2_files/figure-latex/unnamed-chunk-22-1.pdf}
In all of the simple linear regression models, we observe strong
evidence of statistical significance for each predictor, with p-values
close to 0. This suggests that, when considered individually, each
predictor has a strong association with per capita crime rate (crim).

However, this is expected because each model only includes one predictor
at a time, meaning the regression is not accounting for interactions or
correlations between multiple predictors. Some of these relationships
may weaken when we include multiple predictors in a multiple regression
model, due to multicollinearity or shared explanatory power.

\emph{(b) Fit a multiple regression model to predict the response using
all of the predictors. Describe your results. For which predictors can
we reject the null hypothesis H0 : βj = 0? }

\begin{Shaded}
\begin{Highlighting}[]
\NormalTok{multi\_model }\OtherTok{\textless{}{-}} \FunctionTok{lm}\NormalTok{(crim }\SpecialCharTok{\textasciitilde{}}\NormalTok{ ., }\AttributeTok{data =}\NormalTok{ Boston)}
\FunctionTok{summary}\NormalTok{(multi\_model)}
\end{Highlighting}
\end{Shaded}

\begin{verbatim}
## 
## Call:
## lm(formula = crim ~ ., data = Boston)
## 
## Residuals:
##    Min     1Q Median     3Q    Max 
## -8.534 -2.248 -0.348  1.087 73.923 
## 
## Coefficients:
##               Estimate Std. Error t value Pr(>|t|)    
## (Intercept) 13.7783938  7.0818258   1.946 0.052271 .  
## zn           0.0457100  0.0187903   2.433 0.015344 *  
## indus       -0.0583501  0.0836351  -0.698 0.485709    
## chas1       -0.8253776  1.1833963  -0.697 0.485841    
## nox         -9.9575865  5.2898242  -1.882 0.060370 .  
## rm           0.6289107  0.6070924   1.036 0.300738    
## age         -0.0008483  0.0179482  -0.047 0.962323    
## dis         -1.0122467  0.2824676  -3.584 0.000373 ***
## rad          0.6124653  0.0875358   6.997 8.59e-12 ***
## tax         -0.0037756  0.0051723  -0.730 0.465757    
## ptratio     -0.3040728  0.1863598  -1.632 0.103393    
## lstat        0.1388006  0.0757213   1.833 0.067398 .  
## medv        -0.2200564  0.0598240  -3.678 0.000261 ***
## ---
## Signif. codes:  0 '***' 0.001 '**' 0.01 '*' 0.05 '.' 0.1 ' ' 1
## 
## Residual standard error: 6.46 on 493 degrees of freedom
## Multiple R-squared:  0.4493, Adjusted R-squared:  0.4359 
## F-statistic: 33.52 on 12 and 493 DF,  p-value: < 2.2e-16
\end{verbatim}

In the multiple regression model, the number of statistically
significant predictors is drastically reduced compared to the simple
regression models. In this case, only dis (distance to employment
centers), rad (accessibility to highways), and medv (median home value)
remain statistically significant, with p-values close to 0.

This suggests that we can confidently reject the null hypothesis,
\(H_0 : \beta_j = 0\) for these predictors at all conventional
confidence levels.

\emph{(c) How do your results from (a) compare to your results from (b)?
Create a plot displaying the univariate regression coefficients from (a)
on the x-axis, and the multiple regression coefficients from (b) on the
y-axis. That is, each predictor is displayed as a single point in the
plot. Its coefficient in a simple linear regression model is shown on
the x-axis, and its coefficient estimate in the multiple linear
regression model is shown on the y-axis.}

\begin{Shaded}
\begin{Highlighting}[]
\NormalTok{simple\_coefficients }\OtherTok{\textless{}{-}} \FunctionTok{sapply}\NormalTok{(}\FunctionTok{names}\NormalTok{(Boston)[}\SpecialCharTok{{-}}\DecValTok{1}\NormalTok{], }\ControlFlowTok{function}\NormalTok{(var) \{}
\NormalTok{  model }\OtherTok{\textless{}{-}} \FunctionTok{lm}\NormalTok{(}\FunctionTok{as.formula}\NormalTok{(}\FunctionTok{paste}\NormalTok{(}\StringTok{"crim \textasciitilde{}"}\NormalTok{, var)), }\AttributeTok{data =}\NormalTok{ Boston)  }\CommentTok{\# Fit model}
  \FunctionTok{coef}\NormalTok{(model)[}\DecValTok{2}\NormalTok{]  }\CommentTok{\# Extract the coefficient (slope)}
\NormalTok{\})}
\NormalTok{multi\_coefficients }\OtherTok{\textless{}{-}} \FunctionTok{coef}\NormalTok{(multi\_model)[}\SpecialCharTok{{-}}\DecValTok{1}\NormalTok{] }
\FunctionTok{plot}\NormalTok{(simple\_coefficients, multi\_coefficients, }
     \AttributeTok{xlab =} \StringTok{"Simple Reg. Coeff."}\NormalTok{, }
     \AttributeTok{ylab =} \StringTok{"Multi. Reg. Coeff."}\NormalTok{, }
     \AttributeTok{main =} \StringTok{"Simple vs Multiple Reg. Coefficients"}\NormalTok{, }
     \AttributeTok{pch =} \DecValTok{19}\NormalTok{, }\AttributeTok{col =} \StringTok{"blue"}\NormalTok{)}

\FunctionTok{abline}\NormalTok{(}\AttributeTok{a =} \DecValTok{0}\NormalTok{, }\AttributeTok{b =} \DecValTok{1}\NormalTok{, }\AttributeTok{col =} \StringTok{"red"}\NormalTok{, }\AttributeTok{lwd =} \DecValTok{2}\NormalTok{) }
\end{Highlighting}
\end{Shaded}

\includegraphics{Econ573_ProblemSet2_Part2_files/figure-latex/unnamed-chunk-24-1.pdf}

\begin{Shaded}
\begin{Highlighting}[]
\NormalTok{coefficient\_differences }\OtherTok{\textless{}{-}}\NormalTok{ simple\_coefficients }\SpecialCharTok{{-}}\NormalTok{ multi\_coefficients}
\NormalTok{sorted\_differences }\OtherTok{\textless{}{-}} \FunctionTok{sort}\NormalTok{(coefficient\_differences, }\AttributeTok{decreasing =} \ConstantTok{TRUE}\NormalTok{)}
\FunctionTok{print}\NormalTok{(sorted\_differences)}
\end{Highlighting}
\end{Shaded}

\begin{verbatim}
##         nox.nox ptratio.ptratio     indus.indus     lstat.lstat         age.age 
##    41.206117748     1.456055544     0.568126442     0.410004185     0.108634506 
##         tax.tax         rad.rad           zn.zn       medv.medv         dis.dis 
##     0.033517899     0.005445616    -0.119645016    -0.143103563    -0.538654944 
##      chas.chas1           rm.rm 
##    -1.067398999    -3.312961886
\end{verbatim}

\begin{Shaded}
\begin{Highlighting}[]
\NormalTok{largest\_changes }\OtherTok{\textless{}{-}} \FunctionTok{sort}\NormalTok{(}\FunctionTok{abs}\NormalTok{(coefficient\_differences), }\AttributeTok{decreasing =} \ConstantTok{TRUE}\NormalTok{)}
\FunctionTok{print}\NormalTok{(largest\_changes)}
\end{Highlighting}
\end{Shaded}

\begin{verbatim}
##         nox.nox           rm.rm ptratio.ptratio      chas.chas1     indus.indus 
##    41.206117748     3.312961886     1.456055544     1.067398999     0.568126442 
##         dis.dis     lstat.lstat       medv.medv           zn.zn         age.age 
##     0.538654944     0.410004185     0.143103563     0.119645016     0.108634506 
##         tax.tax         rad.rad 
##     0.033517899     0.005445616
\end{verbatim}

The comparison between simple and multiple regression coefficients
reveals that predictors like nox and rm experience substantial changes,
with their coefficients drastically decreasing in the multiple
regression model. This suggests these predictors are highly correlated
with others and their effects are diminished when accounting for
additional variables. In contrast, predictors like lstat, ptratio, and
age remain consistent across both models, indicating they are strong
independent predictors of crim.

\emph{(d) Is there evidence of non-linear association between any of the
predictors and the response? To answer this question, for each predictor
X, ft a model of the form Y = β0 + β1X + β2X2 + β3X3 + ϵ.}

\begin{Shaded}
\begin{Highlighting}[]
\CommentTok{\# Fit polynomial regression models for each predictor}
\NormalTok{model\_zn }\OtherTok{\textless{}{-}} \FunctionTok{lm}\NormalTok{(crim }\SpecialCharTok{\textasciitilde{}} \FunctionTok{poly}\NormalTok{(zn, }\DecValTok{3}\NormalTok{), }\AttributeTok{data =}\NormalTok{ Boston)}
\NormalTok{model\_indus }\OtherTok{\textless{}{-}} \FunctionTok{lm}\NormalTok{(crim }\SpecialCharTok{\textasciitilde{}} \FunctionTok{poly}\NormalTok{(indus, }\DecValTok{3}\NormalTok{), }\AttributeTok{data =}\NormalTok{ Boston)}
\NormalTok{model\_nox }\OtherTok{\textless{}{-}} \FunctionTok{lm}\NormalTok{(crim }\SpecialCharTok{\textasciitilde{}} \FunctionTok{poly}\NormalTok{(nox, }\DecValTok{3}\NormalTok{), }\AttributeTok{data =}\NormalTok{ Boston)}
\NormalTok{model\_rm }\OtherTok{\textless{}{-}} \FunctionTok{lm}\NormalTok{(crim }\SpecialCharTok{\textasciitilde{}} \FunctionTok{poly}\NormalTok{(rm, }\DecValTok{3}\NormalTok{), }\AttributeTok{data =}\NormalTok{ Boston)}
\NormalTok{model\_age }\OtherTok{\textless{}{-}} \FunctionTok{lm}\NormalTok{(crim }\SpecialCharTok{\textasciitilde{}} \FunctionTok{poly}\NormalTok{(age, }\DecValTok{3}\NormalTok{), }\AttributeTok{data =}\NormalTok{ Boston)}
\NormalTok{model\_dis }\OtherTok{\textless{}{-}} \FunctionTok{lm}\NormalTok{(crim }\SpecialCharTok{\textasciitilde{}} \FunctionTok{poly}\NormalTok{(dis, }\DecValTok{3}\NormalTok{), }\AttributeTok{data =}\NormalTok{ Boston)}
\NormalTok{model\_rad }\OtherTok{\textless{}{-}} \FunctionTok{lm}\NormalTok{(crim }\SpecialCharTok{\textasciitilde{}} \FunctionTok{poly}\NormalTok{(rad, }\DecValTok{3}\NormalTok{), }\AttributeTok{data =}\NormalTok{ Boston)}
\NormalTok{model\_tax }\OtherTok{\textless{}{-}} \FunctionTok{lm}\NormalTok{(crim }\SpecialCharTok{\textasciitilde{}} \FunctionTok{poly}\NormalTok{(tax, }\DecValTok{3}\NormalTok{), }\AttributeTok{data =}\NormalTok{ Boston)}
\NormalTok{model\_ptratio }\OtherTok{\textless{}{-}} \FunctionTok{lm}\NormalTok{(crim }\SpecialCharTok{\textasciitilde{}} \FunctionTok{poly}\NormalTok{(ptratio, }\DecValTok{3}\NormalTok{), }\AttributeTok{data =}\NormalTok{ Boston)}
\NormalTok{model\_lstat }\OtherTok{\textless{}{-}} \FunctionTok{lm}\NormalTok{(crim }\SpecialCharTok{\textasciitilde{}} \FunctionTok{poly}\NormalTok{(lstat, }\DecValTok{3}\NormalTok{), }\AttributeTok{data =}\NormalTok{ Boston)}
\NormalTok{model\_medv }\OtherTok{\textless{}{-}} \FunctionTok{lm}\NormalTok{(crim }\SpecialCharTok{\textasciitilde{}} \FunctionTok{poly}\NormalTok{(medv, }\DecValTok{3}\NormalTok{), }\AttributeTok{data =}\NormalTok{ Boston)}

\FunctionTok{summary}\NormalTok{(model\_zn)}
\end{Highlighting}
\end{Shaded}

\begin{verbatim}
## 
## Call:
## lm(formula = crim ~ poly(zn, 3), data = Boston)
## 
## Residuals:
##    Min     1Q Median     3Q    Max 
## -4.821 -4.614 -1.294  0.473 84.130 
## 
## Coefficients:
##              Estimate Std. Error t value Pr(>|t|)    
## (Intercept)    3.6135     0.3722   9.709  < 2e-16 ***
## poly(zn, 3)1 -38.7498     8.3722  -4.628  4.7e-06 ***
## poly(zn, 3)2  23.9398     8.3722   2.859  0.00442 ** 
## poly(zn, 3)3 -10.0719     8.3722  -1.203  0.22954    
## ---
## Signif. codes:  0 '***' 0.001 '**' 0.01 '*' 0.05 '.' 0.1 ' ' 1
## 
## Residual standard error: 8.372 on 502 degrees of freedom
## Multiple R-squared:  0.05824,    Adjusted R-squared:  0.05261 
## F-statistic: 10.35 on 3 and 502 DF,  p-value: 1.281e-06
\end{verbatim}

\begin{Shaded}
\begin{Highlighting}[]
\FunctionTok{summary}\NormalTok{(model\_indus)}
\end{Highlighting}
\end{Shaded}

\begin{verbatim}
## 
## Call:
## lm(formula = crim ~ poly(indus, 3), data = Boston)
## 
## Residuals:
##    Min     1Q Median     3Q    Max 
## -8.278 -2.514  0.054  0.764 79.713 
## 
## Coefficients:
##                 Estimate Std. Error t value Pr(>|t|)    
## (Intercept)        3.614      0.330  10.950  < 2e-16 ***
## poly(indus, 3)1   78.591      7.423  10.587  < 2e-16 ***
## poly(indus, 3)2  -24.395      7.423  -3.286  0.00109 ** 
## poly(indus, 3)3  -54.130      7.423  -7.292  1.2e-12 ***
## ---
## Signif. codes:  0 '***' 0.001 '**' 0.01 '*' 0.05 '.' 0.1 ' ' 1
## 
## Residual standard error: 7.423 on 502 degrees of freedom
## Multiple R-squared:  0.2597, Adjusted R-squared:  0.2552 
## F-statistic: 58.69 on 3 and 502 DF,  p-value: < 2.2e-16
\end{verbatim}

\begin{Shaded}
\begin{Highlighting}[]
\FunctionTok{summary}\NormalTok{(model\_nox)}
\end{Highlighting}
\end{Shaded}

\begin{verbatim}
## 
## Call:
## lm(formula = crim ~ poly(nox, 3), data = Boston)
## 
## Residuals:
##    Min     1Q Median     3Q    Max 
## -9.110 -2.068 -0.255  0.739 78.302 
## 
## Coefficients:
##               Estimate Std. Error t value Pr(>|t|)    
## (Intercept)     3.6135     0.3216  11.237  < 2e-16 ***
## poly(nox, 3)1  81.3720     7.2336  11.249  < 2e-16 ***
## poly(nox, 3)2 -28.8286     7.2336  -3.985 7.74e-05 ***
## poly(nox, 3)3 -60.3619     7.2336  -8.345 6.96e-16 ***
## ---
## Signif. codes:  0 '***' 0.001 '**' 0.01 '*' 0.05 '.' 0.1 ' ' 1
## 
## Residual standard error: 7.234 on 502 degrees of freedom
## Multiple R-squared:  0.297,  Adjusted R-squared:  0.2928 
## F-statistic: 70.69 on 3 and 502 DF,  p-value: < 2.2e-16
\end{verbatim}

\begin{Shaded}
\begin{Highlighting}[]
\FunctionTok{summary}\NormalTok{(model\_rm)}
\end{Highlighting}
\end{Shaded}

\begin{verbatim}
## 
## Call:
## lm(formula = crim ~ poly(rm, 3), data = Boston)
## 
## Residuals:
##     Min      1Q  Median      3Q     Max 
## -18.485  -3.468  -2.221  -0.015  87.219 
## 
## Coefficients:
##              Estimate Std. Error t value Pr(>|t|)    
## (Intercept)    3.6135     0.3703   9.758  < 2e-16 ***
## poly(rm, 3)1 -42.3794     8.3297  -5.088 5.13e-07 ***
## poly(rm, 3)2  26.5768     8.3297   3.191  0.00151 ** 
## poly(rm, 3)3  -5.5103     8.3297  -0.662  0.50858    
## ---
## Signif. codes:  0 '***' 0.001 '**' 0.01 '*' 0.05 '.' 0.1 ' ' 1
## 
## Residual standard error: 8.33 on 502 degrees of freedom
## Multiple R-squared:  0.06779,    Adjusted R-squared:  0.06222 
## F-statistic: 12.17 on 3 and 502 DF,  p-value: 1.067e-07
\end{verbatim}

\begin{Shaded}
\begin{Highlighting}[]
\FunctionTok{summary}\NormalTok{(model\_age)}
\end{Highlighting}
\end{Shaded}

\begin{verbatim}
## 
## Call:
## lm(formula = crim ~ poly(age, 3), data = Boston)
## 
## Residuals:
##    Min     1Q Median     3Q    Max 
## -9.762 -2.673 -0.516  0.019 82.842 
## 
## Coefficients:
##               Estimate Std. Error t value Pr(>|t|)    
## (Intercept)     3.6135     0.3485  10.368  < 2e-16 ***
## poly(age, 3)1  68.1820     7.8397   8.697  < 2e-16 ***
## poly(age, 3)2  37.4845     7.8397   4.781 2.29e-06 ***
## poly(age, 3)3  21.3532     7.8397   2.724  0.00668 ** 
## ---
## Signif. codes:  0 '***' 0.001 '**' 0.01 '*' 0.05 '.' 0.1 ' ' 1
## 
## Residual standard error: 7.84 on 502 degrees of freedom
## Multiple R-squared:  0.1742, Adjusted R-squared:  0.1693 
## F-statistic: 35.31 on 3 and 502 DF,  p-value: < 2.2e-16
\end{verbatim}

\begin{Shaded}
\begin{Highlighting}[]
\FunctionTok{summary}\NormalTok{(model\_dis)}
\end{Highlighting}
\end{Shaded}

\begin{verbatim}
## 
## Call:
## lm(formula = crim ~ poly(dis, 3), data = Boston)
## 
## Residuals:
##     Min      1Q  Median      3Q     Max 
## -10.757  -2.588   0.031   1.267  76.378 
## 
## Coefficients:
##               Estimate Std. Error t value Pr(>|t|)    
## (Intercept)     3.6135     0.3259  11.087  < 2e-16 ***
## poly(dis, 3)1 -73.3886     7.3315 -10.010  < 2e-16 ***
## poly(dis, 3)2  56.3730     7.3315   7.689 7.87e-14 ***
## poly(dis, 3)3 -42.6219     7.3315  -5.814 1.09e-08 ***
## ---
## Signif. codes:  0 '***' 0.001 '**' 0.01 '*' 0.05 '.' 0.1 ' ' 1
## 
## Residual standard error: 7.331 on 502 degrees of freedom
## Multiple R-squared:  0.2778, Adjusted R-squared:  0.2735 
## F-statistic: 64.37 on 3 and 502 DF,  p-value: < 2.2e-16
\end{verbatim}

\begin{Shaded}
\begin{Highlighting}[]
\FunctionTok{summary}\NormalTok{(model\_rad)}
\end{Highlighting}
\end{Shaded}

\begin{verbatim}
## 
## Call:
## lm(formula = crim ~ poly(rad, 3), data = Boston)
## 
## Residuals:
##     Min      1Q  Median      3Q     Max 
## -10.381  -0.412  -0.269   0.179  76.217 
## 
## Coefficients:
##               Estimate Std. Error t value Pr(>|t|)    
## (Intercept)     3.6135     0.2971  12.164  < 2e-16 ***
## poly(rad, 3)1 120.9074     6.6824  18.093  < 2e-16 ***
## poly(rad, 3)2  17.4923     6.6824   2.618  0.00912 ** 
## poly(rad, 3)3   4.6985     6.6824   0.703  0.48231    
## ---
## Signif. codes:  0 '***' 0.001 '**' 0.01 '*' 0.05 '.' 0.1 ' ' 1
## 
## Residual standard error: 6.682 on 502 degrees of freedom
## Multiple R-squared:    0.4,  Adjusted R-squared:  0.3965 
## F-statistic: 111.6 on 3 and 502 DF,  p-value: < 2.2e-16
\end{verbatim}

\begin{Shaded}
\begin{Highlighting}[]
\FunctionTok{summary}\NormalTok{(model\_tax)}
\end{Highlighting}
\end{Shaded}

\begin{verbatim}
## 
## Call:
## lm(formula = crim ~ poly(tax, 3), data = Boston)
## 
## Residuals:
##     Min      1Q  Median      3Q     Max 
## -13.273  -1.389   0.046   0.536  76.950 
## 
## Coefficients:
##               Estimate Std. Error t value Pr(>|t|)    
## (Intercept)     3.6135     0.3047  11.860  < 2e-16 ***
## poly(tax, 3)1 112.6458     6.8537  16.436  < 2e-16 ***
## poly(tax, 3)2  32.0873     6.8537   4.682 3.67e-06 ***
## poly(tax, 3)3  -7.9968     6.8537  -1.167    0.244    
## ---
## Signif. codes:  0 '***' 0.001 '**' 0.01 '*' 0.05 '.' 0.1 ' ' 1
## 
## Residual standard error: 6.854 on 502 degrees of freedom
## Multiple R-squared:  0.3689, Adjusted R-squared:  0.3651 
## F-statistic:  97.8 on 3 and 502 DF,  p-value: < 2.2e-16
\end{verbatim}

\begin{Shaded}
\begin{Highlighting}[]
\FunctionTok{summary}\NormalTok{(model\_ptratio)}
\end{Highlighting}
\end{Shaded}

\begin{verbatim}
## 
## Call:
## lm(formula = crim ~ poly(ptratio, 3), data = Boston)
## 
## Residuals:
##    Min     1Q Median     3Q    Max 
## -6.833 -4.146 -1.655  1.408 82.697 
## 
## Coefficients:
##                   Estimate Std. Error t value Pr(>|t|)    
## (Intercept)          3.614      0.361  10.008  < 2e-16 ***
## poly(ptratio, 3)1   56.045      8.122   6.901 1.57e-11 ***
## poly(ptratio, 3)2   24.775      8.122   3.050  0.00241 ** 
## poly(ptratio, 3)3  -22.280      8.122  -2.743  0.00630 ** 
## ---
## Signif. codes:  0 '***' 0.001 '**' 0.01 '*' 0.05 '.' 0.1 ' ' 1
## 
## Residual standard error: 8.122 on 502 degrees of freedom
## Multiple R-squared:  0.1138, Adjusted R-squared:  0.1085 
## F-statistic: 21.48 on 3 and 502 DF,  p-value: 4.171e-13
\end{verbatim}

\begin{Shaded}
\begin{Highlighting}[]
\FunctionTok{summary}\NormalTok{(model\_lstat)}
\end{Highlighting}
\end{Shaded}

\begin{verbatim}
## 
## Call:
## lm(formula = crim ~ poly(lstat, 3), data = Boston)
## 
## Residuals:
##     Min      1Q  Median      3Q     Max 
## -15.234  -2.151  -0.486   0.066  83.353 
## 
## Coefficients:
##                 Estimate Std. Error t value Pr(>|t|)    
## (Intercept)       3.6135     0.3392  10.654   <2e-16 ***
## poly(lstat, 3)1  88.0697     7.6294  11.543   <2e-16 ***
## poly(lstat, 3)2  15.8882     7.6294   2.082   0.0378 *  
## poly(lstat, 3)3 -11.5740     7.6294  -1.517   0.1299    
## ---
## Signif. codes:  0 '***' 0.001 '**' 0.01 '*' 0.05 '.' 0.1 ' ' 1
## 
## Residual standard error: 7.629 on 502 degrees of freedom
## Multiple R-squared:  0.2179, Adjusted R-squared:  0.2133 
## F-statistic: 46.63 on 3 and 502 DF,  p-value: < 2.2e-16
\end{verbatim}

\begin{Shaded}
\begin{Highlighting}[]
\FunctionTok{summary}\NormalTok{(model\_medv)}
\end{Highlighting}
\end{Shaded}

\begin{verbatim}
## 
## Call:
## lm(formula = crim ~ poly(medv, 3), data = Boston)
## 
## Residuals:
##     Min      1Q  Median      3Q     Max 
## -24.427  -1.976  -0.437   0.439  73.655 
## 
## Coefficients:
##                Estimate Std. Error t value Pr(>|t|)    
## (Intercept)       3.614      0.292  12.374  < 2e-16 ***
## poly(medv, 3)1  -75.058      6.569 -11.426  < 2e-16 ***
## poly(medv, 3)2   88.086      6.569  13.409  < 2e-16 ***
## poly(medv, 3)3  -48.033      6.569  -7.312 1.05e-12 ***
## ---
## Signif. codes:  0 '***' 0.001 '**' 0.01 '*' 0.05 '.' 0.1 ' ' 1
## 
## Residual standard error: 6.569 on 502 degrees of freedom
## Multiple R-squared:  0.4202, Adjusted R-squared:  0.4167 
## F-statistic: 121.3 on 3 and 502 DF,  p-value: < 2.2e-16
\end{verbatim}

The results reveal strong evidence of non-linear relationships between
several predictors and crime rate (crim). Notably, nox (nitric oxide
concentration), dis (distance to employment centers), and medv (median
home value) exhibit highly significant quadratic and cubic terms, with
p-values close to 0. This indicates complex non-linear associations that
impact crime rate. Predictors like indus (non-retail business acres),
age (proportion of older housing), and ptratio (pupil-teacher ratio)
show moderate non-linear effects with significant quadratic and cubic
terms with their p-values being less than 0.01

In contrast, predictors such as zn (residential land zoning), rad
(accessibility to highways), and tax (property tax rate) demonstrate
weak or no significant non-linear effects, suggesting simpler models may
suffice for them. In conclusion, we could incorporate non-linear terms
for key predictors like nox, dis, and medv to enhance our model.

\end{document}
